\noindent Suppose that $\mathscr{L}(\phi_a, \partial_\mu \phi_a)$ is the Lagrangian density for some set of fields $\{ \phi_a(x) \}_{a=1}^N$. Recall that the Lagrangian density is a compact, encrypted of writing the equations of motion of a system, and the Euler-Lagrange equations and the principle of least action are used to unpack/decrypt the equations of motion. \\

\noindent Now consider an infinitesimal continuous transformation of the fields
\begin{equation}
\phi_a '(x) = \phi_a(x) + X_a(\phi_a)
\end{equation}

\noindent This produces an \textit{infinitesimal symmetry} in the equations of motion when they are left invariant under the principle of least action
\begin{align}
\mathscr{L} &\rightarrow \mathscr{L}(\phi_a', \partial_\mu \phi_a') = \mathscr{L}(\phi_a, \partial_\mu \phi_a) + \partial_\mu F^\mu \\
\& \,\,\,\, \mathcal{S} [ \phi_a ] &= \int d^4 x \mathscr{L} = \int d^4 x ( \mathscr{L} + \partial_\mu F^\mu )
\end{align}

\subsection*{Noether's Theorem}

\noindent Every continuous symmetry of a Lagrangian implies the existence of a conserved current $j^\mu(x)$. \\

\noindent \textbf{Proof:} Let $X_a[\phi_a] = \delta \phi_a$ be an arbitrary, infinitesimal change in each field, such that the infinitesimal change in the Lagrangian density is 

\begin{align}
\delta \mathscr{L}(\phi_a, \partial_\mu \phi_a) &= \mathscr{L}(\phi_a+\delta \phi_a, \partial_\mu(\phi_a + \delta \phi_a)) - \mathscr{L}(\phi_a, \delta_\mu \phi_a) \\
&\cong \frac{\partial \mathscr{L}}{\partial \phi_a} \delta \phi_a + \frac{\partial \mathscr{L}}{\partial (\partial_\mu \phi_a)} \delta(\partial_\mu \phi_a) \\
&= \left( \frac{\partial \mathscr{L}}{\partial \phi_a} - \partial_\mu \left( \frac{\partial \mathscr{L}}{\partial (\partial_\mu \phi_a)} \right) \right) \delta \phi_a + \partial_\mu \left( \frac{\partial \mathscr{L}}{\partial (\partial_\mu \phi_a)} \delta \phi_a \right) \\
\delta \mathscr{L}(\phi_a, \partial_\mu \phi_a) &= \partial_\mu \left( \frac{\partial \mathscr{L}}{\partial (\partial_\mu \phi_a)} \delta \phi_a \right)
\end{align}

\noindent Where in line $(14)$, we have kept only first order terms $\mathcal{O} (\delta \phi_a)$, and have used the Taylor expansion and added zero to get to line $(15)$. To obtain line $(16)$, note that the first term in line $(15)$ is equal to zero, since $\phi_a(x)$ obey the Euler-Lagrange equations. \\

\noindent As defined above, for an infinitesimal transformation, call $\delta \phi_a = X_a[\phi_a]$  and  for an infinitesimal symmetry, call $\delta \mathscr{L} = \partial_\mu F^\mu$.

\begin{align}
\partial_\mu F^\mu &= \partial_\mu \left( \frac{\partial \mathscr{L}}{\partial (\partial_\mu \phi_a)} X_a[\phi_a] \right) \\
0 &= \partial_\mu \left( \frac{\partial \mathscr{L}}{\partial (\partial_\mu \phi_a)} X_a[\phi_a] - F^\mu \right)
\end{align}

\noindent Call the conserved quantity the conserved current 
\begin{equation}
j^\mu (x) = \left( \frac{\partial \mathscr{L}}{\partial (\partial_\mu \phi_a)} X_a[\phi_a] - F^\mu \right).
\end{equation}

\noindent Now, a conserved current implies the existence of a conserved charge. For any measurable region in our Minkowski space $V \subset \mathcal{M}$, define the integral of the time-like component of the current as
\begin{equation}
Q_V = \int_V d^3 x \,\, j^0(x)
\end{equation}

\noindent Take $V=\mathbb{R}^3$, and assume that the current vanishes at infinity, such that $j \rightarrow 0$ on the boundary $\partial V$, and take the time derivative
\begin{align}
\frac{d Q_{\mathbb{R}^3}}{d t} &= \int_{\mathbb{R}^3} d^3x \,\, \partial_0 j^0 (x) \\
&= - \int_{\mathbb{R}^3} d^3x \,\, \partial_k j^k \\
&= - \int_{\partial \mathbb{R}^3} j  ds \\
\frac{d Q_{\mathbb{R}^3}}{d t} &= 0
\end{align}

\subsection*{Example}

\noindent Consider an active transformation of spacetime coordinates $x^\mu \rightarrow x^\mu - \epsilon^\mu$.

\noindent Then each field transforms as
\begin{align}
\phi'_a(x^\mu) &= \phi_a(x^\mu + \epsilon^\mu) \\
&= \phi_a(x^\mu) + \epsilon^\nu \partial_\nu \phi_a(x^\mu)
\end{align}

\noindent And the Lagrangian density transforms as, yielding $4 \times 4=16$ equations from summing over $\mu$ and $\nu$ 
\begin{align}
\mathscr{L}(x'^\mu) &= \mathscr{L} (x^\mu + \epsilon^\mu) \\
&= \mathscr{L}(x^\mu) + \epsilon^\nu \partial_\nu \mathscr{L}(x^\mu)
\end{align}

\noindent Where the infinitesimal field transformation is $X_a[\phi_a] = \epsilon^\nu \partial_\nu \phi_a(x^\mu)$, and the infinitesimal symmetry of the Lagrangian density is $\partial_\mu F^\mu = \epsilon^\mu \partial_\mu \mathscr{L}(x^\mu)$ \\

\noindent Consider the infinitesimal element $\epsilon$ with basis vector entries $[\hat{\nu}]^\mu_{\,\,\,\nu} = \delta^\mu_{\,\,\,\nu}$
\begin{equation}
\epsilon^\mu = \epsilon \hat{\nu}^\mu = \epsilon \{ (1,0,0,0), (0,1,0,0), (0,0,1,0), (0,0,0,1) \}.
\end{equation}

\noindent Apply Noether's theorem to each of the 4 symmetry terms $\epsilon \hat{\nu}^\mu$, yielding 16 total terms that we assign as the elements $T^\mu_{\,\,\,\nu}$ of the \textit{energy-momentum} or \textit{stress-energy tensor}
\begin{align}
T^\mu_{\,\,\,\nu} = j^\mu_{\,\,\,\nu} = \frac{\partial \mathscr{L}}{\partial (\partial_\mu \phi_a)} \partial_\nu \phi_a - \delta^\mu_{\,\,\,\nu} \mathscr{L}.
\end{align}

\noindent Note that each of the $\nu^{th}$ columns of the energy-momentum tensor correspond to one of the four conserved currents and translation in each of the $\nu^{th}$ directions
\begin{align}
\partial_\mu T^\mu_{\,\,\,\nu} = 0, \,\,\,\, \forall \,\, \nu.
\end{align}

\noindent In a \textit{closed} system, the corresponding conserved charges, from the columns (conserved currents) of the energy-momentum tensor, are the total energy and the momentum in each of the three spatial directions
\begin{align}
E &= \int d^3x \,\,\, T^{00} \\
p^j &= \int d^3x \,\,\, T^{0j} .
\end{align}

\subsubsection*{Example of the Example: Klein-Gordon Field}

Consider the Klein-Gordon Lagrangian density $\mathscr{L} = \frac{1}{2} \partial_\mu \phi \partial^\mu \phi - \frac{1}{2}m^2 \phi^2$. The energy-momentum tensor has elements of the form
\begin{align}
T^\mu_{\,\,\,\nu} &= \partial^\mu \phi \partial_\nu \phi - \delta^\mu_{\,\,\,\nu}\mathscr{L} \\
T^{\mu \nu} &= \eta^{\nu \nu'} T^\mu_{\,\,\,\nu'} = \partial^\mu \phi \partial^\nu \phi - \eta^{\mu \nu} \mathscr{L}.
\end{align}

\noindent The corresponding conserved charges are
\begin{align}
E &= \int d^3x \,\, \mathscr{H}(x) = \int d^3x \,\, (\partial^0 \phi \partial^0 \phi - \frac{1}{2}\partial^0 \phi \partial^0 \phi + \frac{1}{2} m^2 \phi^2) \\
p^j &= \int d^3x \,\, \dot{\phi} \partial^j \phi .
\end{align}

\subsubsection*{How To Apply Noether's Theorem}
\begin{enumerate}
\item Identify the continuous symmetry. \\
\item Calculate the change in the Lagrangian density. \\
\item Calculate the change in each of the fields. \\
\item Work out the conserved currents and charges.
\end{enumerate}

\subsection*{Infinitesimal Lorentz Transformations}

\noindent Consider the transformation $x^\mu \rightarrow \Lambda^\mu_{\,\,\,\nu} x^\nu$, where $\Lambda^\mu_{\,\,\,\nu} = \delta^\mu_{\,\,\,\nu} + \omega^\mu_{\,\,\,\nu}$, and $\omega^\mu_{\,\,\,\nu}$ is infinitesimal. Next, recall the following property of the group of Lorentz transformations, restricting the possible values for $\omega$
\begin{align}
\eta &= \Lambda^T \eta \Lambda \\
\eta^{\mu \nu} &= (\delta^\mu_{\,\,\,\sigma} + \omega^\mu_{\,\,\,\sigma} ) (\delta^\nu_{\,\,\,\tau} + \omega^\nu_{\,\,\,\tau} ) \eta^{\sigma \tau} \\
0 &=_{\mathcal{O}(\omega)} \omega^{\mu \nu} + \omega^{\nu \mu}
\end{align}

\noindent This is a linear equation in $\omega$, since we have kept only up to first order terms in $\omega$, and tells us that $\omega$ is an \textit{antisymmetric}, infinitesimal generator of Lorentz transformations with six independent variables which define six continuous symmetries and six conserved currents and charges.

\begin{equation}
\omega = \left(
\begin{array}{cccc}
0 & -\alpha & -\beta & -\gamma \\
\alpha & 0 & -\delta & -\epsilon \\
\beta & \delta & 0 & -\kappa \\
\gamma & \epsilon & \kappa & 0 \\
\end{array}
\right)
\end{equation}

\noindent The action of this infinitesimal Lorentz transformation on the fields is 
\begin{align}
\phi_a(x) \rightarrow \phi'_a(x) &= \phi_a(\Lambda^{-1}x) \\
&= \phi_a((\delta-\omega)x) \\
&= \phi_a(x^\mu - \omega^\mu_{\,\,\,\nu} x^\nu) \\ 
&=_{\mathcal{O}(\omega)} \phi_a(x) - \omega^\mu_{\,\,\,\nu} x^\nu \partial_\mu \phi_a(x)
\end{align}

\noindent Showing that the symmetry is defined by the infinitesimals
\begin{align}
\delta \phi_a &= -\omega^\mu_{\,\,\,\nu} x^\nu \partial_\mu \phi_a \\
\& \,\,\,\, \delta \mathscr{L} &= -\omega^\mu_{\,\,\,\nu} x^\nu \partial_\mu \mathscr{L} = -\partial_\mu(\omega^\mu_{\,\,\,\nu} x^\nu \mathscr{L}) \\
\& \,\,\,\, j^\mu_{\,\,\,\omega} &= \frac{\partial \mathscr{L}}{\partial (\partial_\mu \phi_a)} \omega^\rho_{\,\,\,\nu} x^\nu \partial_\rho \phi_a + \omega^\mu_{\,\,\,\nu} x^\nu \mathscr{L}.
\end{align}

\noindent (CHECK how $j$ to $\mathcal{J}$) Applying Noether's theorem tells us that the six independent conserved currents $\partial_\mu (\mathcal{J}^\mu)^{\rho \sigma} = 0$, and conserved charges, are of the form
\begin{align}
(\mathcal{J}^\mu)^{\rho \sigma} &= x^\rho T^{\mu \sigma} - x^\sigma T^{\mu \rho} \\ 
Q^{jk} &= \int d^3x \,\, (x^j T^{0k} - x^k T^{0j}) \\
Q^{0j} &= \int d^3x \,\, (x^0 T^{0j} - x^j T^{00})
\end{align}

\noindent Call $Q^{jk}$ the generators of rotations, and $Q^{0j}$ the generators of boosts of the Lorentz transformations.

\subsection*{Generators}

\noindent Let $f$ and $g$ be maps from phase space to the real numbers 
\begin{equation}
f, \, g: \,\, \mathbb{R}^N \times \mathbb{R}^N \rightarrow \mathbb{R}.
\end{equation}

\noindent Define the Poisson bracket with the pairs of canonical coordinates $(q_j, p_j)$
\begin{align}
\{f,g\} &= \sum_{j=1}^N \left( \frac{\partial f}{\partial q_j} \frac{\partial g}{\partial p_j} - \frac{\partial f}{\partial p_j} \frac{\partial g}{\partial q_j} \right) \\
\{f, H \} &= \frac{d f}{d t}
\end{align}

\noindent The field theory version of the Posson bracket is defined with the canonical coordinate pairs $(\phi(x), \pi(x))$
\begin{align}
\{F, G\} &= \int d^3x \,\, \left( \frac{\delta F}{\delta \phi(x)} \frac{\delta G}{\delta \pi(x)} - \frac{\delta F}{\delta \pi(x)} \frac{\delta G}{\delta \phi(x)} \right) \\
\{f, Q^{\rho \sigma}\} &= \frac{\partial f}{\partial s^{\rho \sigma}}
\end{align}

\noindent Where the Poisson bracket of $f$ and the conserved charges $Q$ generate the corresponding symmetry transformations. Conserved cahrges also obey the Lie algebra obeyed by the Poincar\'e group.

\subsection*{Standard Dogma of Quantization}

\noindent Basically, put hats on things

\begin{itemize}
\item Function $f$ on phase space $\rightarrow$ linear operator $\hat{f}$ (observable) on Hilbert space \\
\item Poisson bracket $\{ f,g \} = h \rightarrow$ commutator $[\hat{f}, \hat{g}] = i \hat{h}$ \\
\item Conserved charge $Q^{\rho \sigma} \rightarrow$ conserved charge operator $\hat{Q}^{\rho \sigma}$
\end{itemize}

\noindent Where the conserved charge operators generate the Lorentz transformations on Hilbert space
\begin{equation}
\frac{d \hat{U}}{d s} = i [\hat{U}, \hat{Q}^{\rho \sigma}] \omega_{\rho \sigma}
\end{equation}