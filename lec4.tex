\noindent More than one quantum field theory can have the same classical field theory as an effective model, making field quantization not a well-posed problem. Developing a quantum field theory is therefore built on educated guesses.

\subsection*{Canonical quantization of particles} 
 
\noindent The standard approach of \textit{canonical quantization} is to begin with a classical theory and suppose $n$ classical degrees of freedom, which are used to measure the canonical coordinate pairs, \textit{position} $q_j$ and \textit{momentum} $p_j$, for each degree of freedom, such that the Poisson bracket is defined by $\{q_j, p_k \} = \delta_{jk}$. The total energy of the of the system is measured by the classical \textit{Hamiltonian}, defined by 
\begin{equation}
H = \sum_{j=1}^n \frac{p_j^2}{2 m} + \frac{m}{2} \sum_{j, k=1}^n q_j [\textbf{Q}]_{jk} q_k
\end{equation}

\noindent Where $\textbf{Q}$ is an $n \times n$ symmetric, positive matrix. \\

\noindent For example, consider the quantum harmonic oscillator, and take the naive approach by basically putting hats on everything. This ends up working for field quantization, and yields a unitary representation of the Poincar\'e group. \\

\begin{itemize}
\item \textbf{Canonical coordinates}: $(q_j, p_j) \\ \rightarrow \textbf{Canonical coordinate operators}: (\hat{q}_j, \hat{p}_j)$ \\
\item \textbf{Poisson bracket}: $\{ q_j, p_k \} = \delta_{jk}$ \\ $\rightarrow$ \textbf{Commutator}: $[\hat{q}_j, \hat{p}_k] = i \delta_{jk}$ \\
\item \textbf{Hamiltonian}: $H$, as defined above \\ $\rightarrow \textbf{Hamiltonian operator}: \hat{H} = \sum_{j=1}^n \frac{\hat{p}_j^2}{2 m} + \frac{m}{2} \sum_{j, k=1}^n \hat{q}_j [\textbf{Q}]_{jk} \hat{q}_k$
\end{itemize}

\noindent To diagonalize the Hamiltonian operator, first note that since $\textbf{Q}$ is a symmetric, positive $n \times n$ matrix, there exists an orthogonal matrix $\textbf{O}$ (s.t., $\textbf{O}^T \textbf{O} = \textbf{I}$), such that $\textbf{O} \textbf{Q} \textbf{O}^T = \textbf{D}$, where $\textbf{D}$ is a diagonal matrix, where we call the diagonal elements $\{ \omega_i^2 \}_{i=1}^n$ \\

\noindent Now, transform the canonical coordinates, using the orthogonal matrix, such that the correct commutation relation is still obeyed
\begin{align}
\hat{q}_j &= \sum_{k=1}^n [\textbf{O}]_{jk} \hat{q}'_k \\
\hat{p}_j &= \sum_{k=1}^n [\textbf{O}]_{jk} \hat{p}'_k \\
i \delta_{jk} &= [ \hat{q}'_j, \hat{p}'_k ]
\end{align}

\noindent Using the facts that $\textbf{O}^T = \textbf{O}^{-1}$ and $\textbf{O} \textbf{Q} \textbf{O}^T = \textbf{D}$, the Hamiltonian becomes diagonalized
\begin{align}
\hat{H} &= \sum_{j=1}^n \frac{\hat{p'}_j^2}{2 m} + \frac{m}{2} \sum_{j, k, l, m=1}^n \hat{q'}_l [\textbf{O}^T]_{jl} [\textbf{Q}]_{jk} [\textbf{O}^T]_{km} \hat{q'}_m \\
\hat{H} &= \sum_{j=1}^n \frac{\hat{p'}_j^2}{2 m} + \frac{1}{2} \sum_{k=1}^n \omega_k^2 \hat{q'}_k^2 \\
\hat{H} &= \frac{1}{2} \sum_{k=1}^n \omega_k (\hat{a}_k^\dagger \hat{a}_k + \frac{1}{2})
\end{align}

\noindent Where the annihilation and creation ladder operators that diagonalize the quantum harmonic oscillator Hamiltonian are defined as
\begin{align}
\hat{a}_k &= \sqrt{\frac{m \omega_k}{2}} (\hat{q}'_k + \frac{i}{m \omega_k} \hat{p}'_k) \\
\hat{a}_k^\dagger &= \sqrt{\frac{m \omega_k}{2}} (\hat{q}'_k - \frac{i}{m \omega_k} \hat{p}'_k)
\end{align}

\subsection*{Canonical Quantization of Fields}

\noindent To quantize the Klein-Gordon field, we follow the same seemingly naive approach of putting hats on everything. In this example fo quantizing a field, the continuous variable $x$ is used, in contrast to the discrete labels $j$ in the previous example of the quantum harmonic oscillator

\begin{itemize}
\item \textbf{Canonical coordinates}: $(\phi(x), \pi(x)) \\ \rightarrow \textbf{Canonical coordinate operators}: (\hat{\phi}(x), \hat{\pi}(x))$ \\
\item \textbf{Poisson bracket}: $\{ \phi(x), \pi(y) \} = \delta^{(3)}(x-y)$ \\ $\rightarrow$ \textbf{Commutator}: $[\hat{\phi}(x), \hat{\pi}(y)] = i \delta^{(3)}(x-y)$ \\
	\subitem Note that this is the \textit{equal time Poisson bracket}, such that $(x-y)$ is the spatial three-vector.
	\subitem Also note that this commutator is strange,  as it is comprised of "two self-adjoint operators and something's that not even a function" \\
\item \textbf{Hamiltonian}: $H_{KG} = \frac{1}{2} \int d^3x \,\, \left( \pi^2(x) + (\nabla \phi(x))^2 + \frac{1}{2} m^2 \phi^2(x) \right)$ \\ $\rightarrow \textbf{Hamiltonian operator}: \hat{H}_{KG} = \frac{1}{2} \int d^3x \,\, \left( \hat{\pi}^2(x) + (\nabla \hat{\phi}(x))^2 + \frac{1}{2} m^2 \hat{\phi}^2(x) \right)$
\end{itemize}

\noindent Essentially, replace discrete sums with continuous integrals, by switching to a continuous label $j \rightarrow x$ and a continuous dynamical variable $q_j \rightarrow q_x=\phi(x)$. Solving the quantum Hamiltonian, by analogy of the canonical quantization of particles, should be as simple as creating the analog of the $\textbf{Q}$ matrix and its diagonalization. \\

\noindent  Replacing sums by integrals allows the full diagonalization of $\textbf{Q}$, and, therefore, the full diagonalization of the Hamiltonian $\hat{H} = \hat{H}_{KG}$, but this does not yet yield a unitary representation of the Poincar\'e group or a valid relativistic quantum field theory. Diagonalizing the Hamiltonian only quantizes a one-parameter subgroup of the Poincar\'e group. The conserved currents, charges, and operators obeying the correct Lie algebra are still needed for a relativistic quantum field theory. \\

\subsubsection*{Diagonalization of the quantum field theory}

\noindent The diagonalization of a field theory begins with emergence of the \textit{Fourier transform}. Replace sums with integrals, and, since the matrix elements are described by two numbers, let's define a continuous function in two variables $K(x,y)$

\begin{equation}
\hat{q}_j = \sum_k [\textbf{O}]_{jk} \hat{q}'_k \,\,\,\, \rightarrow \,\,\,\, \hat{\phi}(x) = \int d^3 y \,\, K(x, y) \hat{\phi}(y)
\end{equation}

\noindent Where $K(x,y)$ is the \textit{kernel} of the Fourier transform. \\

\noindent Using the Fourier transform is motivated by certain features of symmetric matrices. Consider the \textit{circulant} matrix, a type of Toeplitz matrix where each successive column is a cyclic permutation of the previous column, initialized by the first column vector, and has the form as an $n \times n$ matrix 

\begin{equation}
\begin{bmatrix} 
 c_0 & c_{n-1} & \dots & c_2 & c_1 \\
 c_1 & c_0 & c_{n-1} &  & c_2 \\
 \vdots & c_1 & c_0 &  \ddots & \vdots \\
 c_{n-2} &  & \ddots &  \ddots & c_{n-1} \\
 c_{n-1} & c_{n-2} & \dots &  c_1 & c_0 
\end{bmatrix}.
\end{equation} \\

\noindent These matrices are diagonalized via the discrete Fourier transform, which is an $n \times n$ unitary matrix, though not orthogonal and may have complex entries

\begin{equation}
\textbf{U} = \frac{1}{\sqrt{n}} 
\begin{bmatrix} 
 1 & 1 & 1 & \dots &  \\
 1 & \mu & \mu^2 & \dots &  \\
 1 & \mu^2 & \mu^4 &   & \vdots \\
 \vdots & \vdots & &  \mu^{jk} &  \\
  &  & \dots &  & \mu^{nn} 
\end{bmatrix}
\end{equation}


\noindent Where $\mu = e^{\frac{2 \pi i}{n}}$ is the $n^{th}$ roots of unity. The elements of the discrete Fourier transform, therefore, have the form $\frac{1}{\sqrt{n}} e^{\frac{2\pi i j k}{n}}$. Compare this to the continuous Fourier transform kernel function $K(x,y) = \frac{1}{2\pi} e^{ixy}$. \\

\noindent For transformations between position and momentum space, make the guess that the Fourier transform that will diagonalize our quantized Klein-Gordon Hamiltonian has the form 

\begin{equation}
\hat{\phi}(x) = \int \frac{d^3 p}{(2\pi)^3} e^{i p \cdot x} \hat{\phi}_p(p).
\end{equation}

\noindent Where $\hat{\phi}_p(p)$ is the momentum space wavefunction, and is not Hermitian, such that $\hat{\phi}_p(p)^\dagger = \hat{\phi}_p(-p)$. To check if the guess is correct, apply $\hat{H}_{KG}$ to the transform defined above, and observe whether it is diagonalized or not. \\

\noindent As in the discrete case of diagonalization, we construct ladder operators

\begin{align}
\hat{\phi}(x) &= \int \frac{d^3 p}{(2 \pi)^3} \frac{1}{\sqrt{2 \omega_p}} \left( \hat{a}_p e^{ip \cdot x} + \hat{a}_p^\dagger e^{-ip \cdot x} \right) \\
\hat{\pi}(x) &=-i  \int \frac{d^3 p}{(2 \pi)^3} \sqrt{\frac{\omega_p}{2}} \left( \hat{a}_p e^{ip \cdot x} - \hat{a}_p^\dagger e^{-ip \cdot x} \right) \\
&\omega_p = \sqrt{ |p|^2 + m^2}
\end{align}

\noindent Check the commutation relation
\begin{align}
[ \hat{\phi}(x), \hat{\pi}(x') ] &= \frac{-i}{2} \int \frac{d^3 p d^3 p'}{(2 \pi)^6} \sqrt{\frac{\omega_{p'}}{\omega_p}} \left( [\hat{a}_{-p}^\dagger, \hat{a}_{p'}] - [\hat{a}_p, \hat{a}_{-p'}^\dagger] \right) e^{i(p \cdot x+p' \cdot y)} \\
& \,\,\,\, \,\,\,\, \,\,\,\, \,\,\,\, \,\,\,\, \left( [\hat{a}_p, \hat{a}_{p'}^\dagger] = (2\pi)^3 \delta^{(3)}(p-p') \cdot \mathbb{I} \right) \\
 &= i \delta^{(3)}(x-y) 
\end{align}

\noindent Making this substitution, the quantum Klein-Gordon Hamiltonian is diagonalized

\begin{align}
\hat{H}_{KG} &= \int d^3 x \int \frac{d^3 p d^3 p'}{(2 \pi)^6} e^{i(p+p') \cdot x} ( \frac{-1}{4} \sqrt{\omega_p \omega_{p'}} (\hat{a}_p - \hat{a}_{-p}^\dagger)(\hat{a}_{p'} - \hat{a}_{-p'}^\dagger) \\
&\,\,\,\, \,\,\,\, + \frac{-pp' + m^2}{4 \sqrt{\omega_p \omega_{p'}}} (\hat{a}_p + \hat{a}_{-p}^\dagger)(\hat{a}_{p'} + \hat{a}_{-p'}^\dagger) ) \\
&= \int \frac{d^3 p}{(2\pi)^3} \omega_p (\hat{a}_p^\dagger \hat{a}_p + \frac{1}{2} [\hat{a}_p, \hat{a}_p^\dagger]) \\
&= \int \frac{d^3 p}{(2\pi)^3} \omega_p (\hat{a}_p^\dagger \hat{a}_p + \frac{1}{2} \delta_{pp} \cdot \mathbb{I}) \\
&\cong \int \frac{d^3 p}{(2\pi)^3} \omega_p \hat{a}_p^\dagger \hat{a}_p
\end{align}

\noindent Where the infinite absolute energy shift is tossed to get the last line, since we only measure energy differences, and $\hat{H}_{KG}$ is diagonalized!